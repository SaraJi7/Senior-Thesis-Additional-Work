\begin{hangparas}{.25in}{1}
Abadie, A., Diamond, A., $\And$ Hainmueller, J. (2015). “Comparative politics and the synthetic control method.” \textit{American Journal of Political Science}, 59(2), 495-510.

Akerlof, G. A., $\And$ Kranton, R. E. (2002). “Identity and schooling: Some lessons for the economics of education.” \textit{Journal of economic literature}, 40(4), 1167-1201.

Alonso, G. (2009). \textit{Culture trap: Talking about young people of color and their education.} Our schools suck: Students talk back to a segregated nation on the failures of urban education, 31-67.

Angrist, J. D., $\And$ Lang, K. (2004). “Does school integration generate peer effects? Evidence from Boston's Metco Program.” \textit{American Economic Review}, 94(5), 1613-1634.

Armor, D. J. (1995). “Can desegregation alone close the achievement gap.” \textit{Education Week}, 14, 41.

Armor, D. J. (2001). “The end of school desegregation and the achievement gap.” \textit{Hastings Constitutional Law Quarterly}, 28(3), 629-654.

Banks, J. A. (1991). “A curriculum for empowerment, action, and change.” \textit{Empowerment through multicultural education}, 125-141.

Baker, B. D. (2012). “Revisiting the age-old question: Does money matter in education?” Washington, DC: Albert Shanker Institute.

Byrd, C. M. (2016). “Does culturally relevant teaching work? An examination from student perspectives.” \textit{Sage Open}, 6(3), 2158244016660744. 

Center for Education Policy Analysis, Stanford University. Racial and Ethnic Achievement Gaps. Retrieved from https://cepa.stanford.edu/educational-opportunity-monitoring-project/achievement-gaps/race/#second

Chavous, T. M., Bernat, D. H., Schmeelk‐Cone, K., Caldwell, C. H., Kohn‐Wood, L., $\And$ Zimmerman, M. A. (2003). “Racial identity and academic attainment among African American adolescents.” \textit{Child development}, 74(4), 1076-1090.

Chetty, Raj, et al. “Zipcode Destiny: The Persistent Power Of Place And Education.” NPR, NPR, 12 Nov. 2018, www.npr.org/2018/11/12/666993130/zipcode-destiny-the-persistent-power-of-place-and-education. 

Ciotti, P. (1998). America’s Most Costly Educational Failure. Retrieved from https://www.cato.org/publications/commentary/americas-most-costly-educational-failure.

Cronin, J., Kingsbury, G. G., McCall, M. S., & Bowe, B. (2005)."The impact of the No Child Left Behind Act on student achievement and growth: 2005 edition." \textit{Portland, OR: Northwest Evaluation Association.}

Dee, T. S., $\And$ Penner, E. K. (2017). “The causal effects of cultural relevance: Evidence from an ethnic studies curriculum.” \textit{American Educational Research Journal}, 54(1), 127-166.

Dee, T., $\And$ Penner, E. (2019). My Brother’s Keeper? The Impact of Targeted Educational Supports (No. w26386). National Bureau of Economic Research.

Deke, J. (2003). “A study of the impact of public school spending on postsecondary educational attainment using statewide school district refinancing in Kansas.” \textit{Economics of Education Review}, 22(3), 275-284.

Deming, D. (2009). “Early childhood intervention and life-cycle skill development: Evidence 
from Head Start.” \textit{American Economic Journal: Applied Economics}, 1(3), 111-34.

Dumas, M. J., $\And$ Nelson, J. D. (2016). “(Re) Imagining Black boyhood: Toward a critical framework for educational research.” \textit{Harvard Educational Review}, 86(1), 27-47.

Gay, G. (1988). “Redesigning relevant curricula for diverse learners.” \textit{Education and Urban Society}, 2(4), 327–340.

Gay, G. (2010). \textit{Culturally responsive teaching: Theory, research, and practice.} Teachers College Press.

Gershenson, S., Hart, C., Hyman, J., Lindsay, C., $\And$ Papageorge, N. W. (2018). The long-run impacts of same-race teachers (No. w25254). National Bureau of Economic Research.

Hansen, M., Levesque, E., Valant, J., &amp; Quintero, D. (2018). "2018 Brown Center report on American education: Trends In NAEP math, reading, and civics scores." Retrieved April 07, 2021, from https://www.brookings.edu/research/2018-brown-center-report-on-american-education-trends-in-naep-math-reading-and-civics-scores/#footnote-7

Hanushek, E. A., Kain, J. F., $\And$ Rivkin, S. G. (2009). “New evidence about Brown v. Board of Education: The complex effects of school racial composition on achievement.” \textit{Journal of labor economics}, 27(3), 349-383.

Hoffer, T., Greeley, A. M., $\And$ Coleman, J. S. (1985). “Achievement growth in public and Catholic schools.” \textit{Sociology of Education}, 74-97.

Howard, T. C. (2013). "How does it feel to be a problem? Black male students, schools, and learning in enhancing the knowledge base to disrupt deficit frameworks." \textit{Review of Research in Education}, 37(1), 54-86.

Hoxby, C. (2000). Peer effects in the classroom: Learning from gender and race variation (No. w7867). National Bureau of Economic Research.

Klivans, L. (2014). “After 4 Years, Oakland Schools' African American Male Achievement Initiative Assesses How It's Doing.” Oakland North. Retrieved April 28, 2021, from oaklandnorth.net/2014/12/17/after-4-years-oakland-schools-african-american-male-achievement-initiative-assesses-how-its-doing/. 

Knudson, J. (2016). “Integrating Academic, Social, and Emotional Learning to Advance Equity and Achievement. Meeting 31 Summary (Oakland, California, December 6-7, 2016).” California Collaborative on District Reform.

Ladson-Billings, G. J. (1992). “Liberatory consequences of literacy: A case of culturally relevant instruction for African American students.” \textit{Journal of Negro Education}, 378–391.

Ladson‐Billings, G. (1995). “But that's just good teaching! The case for culturally relevant pedagogy.” \textit{Theory into practice}, 34(3), 159-165.

Magnuson, K., $\And$ Waldfogel, J. (Eds.). (2008). \textit{Steady gains and stalled progress: Inequality and the Black-White test score gap.} Russell Sage Foundation.

McKown, C., $\And$ Weinstein, R. S. (2008). "Teacher expectations, classroom context, and the achievement gap." \textit{Journal of school psychology}, 46(3), 235-261.

McWhorter, J. (2005).\textit{Winning the race: Beyond the crisis in Black America.}  Penguin.

Nasir, N. S., $\And$ Saxe, G. B. (2003). “Ethnic and academic identities: A cultural practice perspective on emerging tensions and their management in the lives of minority students.” \textit{Educational Researcher}, 32(5), 14–18.

O'connor, C. (1999). “Race, class, and gender in America: Narratives of opportunity among low-income African American youths.” \textit{Sociology of education}, 137-157.

Patterson, O. (2006). "A poverty of the mind". \textit{New York Times}, 26, 27.

Popham, W. J. (1999). "Why standardized tests don't measure educational quality." \textit{Educational leadership}, 56, 8-16.

Reardon, S. F., Weathers, E., Fahle, E., Jang, H., $\And$ Kalogrides, D. (2019). “Is separate still unequal? New evidence on school segregation and racial academic achievement gaps.”

Reardon, S. F. What Explains White-Black Differences in Average Test Scores? Retrieved from https://edopportunity.org/discoveries/white-black-differences-scores/.

Simms, K. (2012). “Is the Black-White Achievement Gap a Public Sector Effect? An Examination of Student Achievement in the Third Grade.” \textit{Journal of At-Risk Issues}, 17(1), 23-29.

Sleeter, C. E. (2012). “Confronting the marginalization of culturally responsive pedagogy.” \textit{Urban Education}, 47(3), 562-584.

Taylor, R. D., Casten, R., Flickinger, S. M., Roberts, D., $\And$ Fulmore, C. D. (1994). “Explaining the school performance of African-American adolescents.” \textit{Journal of Research on Adolescence}, 4(1), 21-44..

Valenzuela, A. (1999). \textit{Subtractive schooling: US-Mexican youth and the politics of caring.} Ithaca, NY: State University of New York Press.

Watson, V. M. (2014). \textit{The black sonrise: Oakland Unified School District’s commitment to address and eliminate institutionalized racism.} Oakland Unified School District. 

Young, E. (2010). “Challenges to conceptualizing and actualizing culturally relevant pedagogy: How viable is the theory in classroom practice?” \textit{Journal of Teacher Education}, 61(3), 248-260.
\end{hangparas}