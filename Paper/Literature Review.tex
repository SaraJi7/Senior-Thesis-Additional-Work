\noindent In the past few decades, policy makers have tried various kinds of ways to close the racial achievement gap. Three of the things that are most commonly tried include racial integration, increased funding, and increased racial awareness. However, evidence of their impact on test scores is mixed.

\noindent\textbf{Racial Integration}

\noindent When it comes to closing the racial achievement gap, one approach is to address racial segregation in schools. Hanushek et al. (2009) finds evidence to support this approach. The study finds that a higher percentage of black schoolmates reduces achievement for blacks, while having a much smaller and generally insignificant effect on whites. Hoxby (2000) also found that racial diversity in the classroom is good for students’ academic achievement, especially for black students. Surprisingly, black, Hispanic, and white third graders all tend to perform worse in reading and math when they are in classes that have a larger share of black students. The more striking result is that the effect of black peers appears to have the greatest negative effect on other black students.

\noindent The black share of school enrollment for black students is usually higher than that of white students. This suggests that distributing black students to white-majority schools is the most effective way to create a more racially diversified school environment. However, the success of such bussing initiatives is achieved by dragging another group of students down. Angrist and Lang (2004) studied the Metco bussing program in the Boston Area and found out that while Metco students’ grades had been improved between third and seventh grades, a higher percentage of black schoolmates reduced achievement for non-Metco black students. Therefore, while bussing might close the gap, it did so by increasing the performance of the black students who were bussed and reducing the performance of those non-Metco black students who became the classmates of the bussed.

\noindent Another example is from Wilmington-New Castle County, Delaware. Via bussing, city students—mostly black—would spend nine years of their K-12 schooling in the suburbs, and suburban students—mostly white—would spend three years in the city. As a result, in all four school districts in the county, both city and suburban schools have been well-integrated since 1976. Despite having some of the most racially integrated schools in the US, the racial achievement gap in these schools is still high and comparable to the national gap (Armor, 2001), indicating the limited effect of school desegregation.

\noindent Even if desegregation could completely close the achievement gap, it’s easier said than done. After the Brown vs. Board of Education decision which banned school segregation, white people moved (for various reasons) to suburban areas too far removed from urban black communities to allow bussing to significantly desegregate the schools. Black students now predominantly attend black urban schools with few resources and white students go to primarily white suburban schools with significantly more funding. 

\noindent\textbf{Increased funding}

\noindent This brings us to the question: what is more important in driving this divergence in educational outcomes, racial segregation or economic segregation? These factors are highly correlated since a much higher proportion of black people have low incomes. A recent study by Reardon et al. (2019) addressed this question. It constructed measures of student performance using test scores in math and English Language Arts (ELA) of students from grade 3-8, from 2009 through 2016, in a variety of school districts nationwide. The regression results show racially segregated classrooms are positively and significantly associated with black under-achievement. The result remained significant even after controlling for socioeconomic status (SES)\footnote{SES indicators include the log of median family income, the proportion of adults with a bachelor’s degree, the poverty rate, the unemployment rate, the SNAP receipt rate, and the single female-headed household rate.} and residential segregation – some of the factors that are commonly regarded as contributing to black students’ underperformance.  However, when the study controlled for racial differences in school poverty, racial segregation no longer had a statistically significant effect. In other words, the relationship between racial segregation and the achievement gap seems to operate through differences in exposure to poor schoolmates. It adversely affects performance because it gathers together black and Hispanic students who are poor themselves, not because of racial segregation per se. This explains why in the Metco bussing program, Angrist and Lang (2004) obtained negative effects of desegregation on non-Metco black students. 

\noindent If school poverty is the problem, would an increased funding help? When we leave race out of the conversation, funding clearly improves achievement. Baker (2012) reviews past empirical studies and concludes that they invariably find a statistically significant positive relationship between student achievement and financial inputs. Baker (2012) tersely concludes, “To be blunt, money does matter.” An example of this would be the Kansas statewide school district finance reforms in the 1990s, which involved primarily a leveling up of low-spending district. According to the estimation by Deke (2003), a 20 percent increase in spending was associated with a 5 percent increase in the likelihood of students going on to postsecondary education.

\noindent However, programs that increase funding for poor black schools have not had much success: Again take Kansas as an example. The Kansas City School District spent nearly 2,000,000,000 dollars on increased funding for all-black schools over 12 years. The district spent money on things like new school buildings, up-to-date materials, state-of-the-art computer labs, increased teacher salaries, and decreased class sizes, however with no noticable improvement in student achievement. Although the district’s 37,000 mostly minority students enjoyed some of the best‐funded school facilities in the country , student performance hadn’t improved. (Ciotti, 1998) After reviewing the evidence of such initiatives, Orfield finds, "the many millions spent have yielded little conclusive evidence that achievement has improved any more than a modest degree." (Armor, 1995)

\noindent\textbf{School Choice}

\noindent Some researchers have proposed private schools as an effective response, as they provide a different form of pedagogy for poor black students as well as potentially more resources (Hoffer, Greeley, and Coleman, 1985). By assigning more homework, longer school hours, and offering more advanced coursework, private school education – primarily Catholic school education – focus more on students’ academic performance. However, more studies find no support for the notion that private school education – primarily Catholic school education – can close the achievement gap for poor black students in racially segregated communities (Lubienski and Lubienski, 2006; Hallinan and Kubitschek, 2012; Simms, 2012).

\noindent Although all above initiatives aim to address the racial achievement gap, none of them has made a great progress.This is because desegregation or money or school choices can’t change the fact that black students are being adultified, criminalized, dehumanized and low-expected (Dumas and Nelson, 2016). Therefore, when Reardon et al. (2019) proposed that if we genuinely want to address racial inequality in educational opportunity, we’d still have to turn back to addressing racial segregation among schools, as “This we do know how to do, or at least we once did”, it has to be noted that this is not and should not be the only option educators have. Researchers have to find out why black students fail.

\noindent\textbf{Increased Racial Awareness}

\noindent This is related to a great misconception of why black students fail academically. People sometimes compare black students with Asian students, and assume that as both groups are minorities, the reason why Asians perform well academically but blacks don’t is because they don’t work hard enough and their cultural values are wrong, and hence it’s their own problem of not succeeding. This is called the Model Minority Myth. Alonso (2009) collected some of these views. He extensively referenced Patterson’s argument on why black students aren’t successful academically. According to Patterson (2006), the sense of “disconnection” that lead to racial achievement gap is “primarily root in African Americans’ cultural values and norms”. Therefore, to address such problem needs a reform of their cultural values and norms rather than traditional government intervention and spending. Although, like McWhorter (2005) explicitly noted in his book Winning the Race, culture does play a huge part in the racial achievement gap, it’s not the reason why black students aren’t succeeding but rather a way that might empower black students and help them to succeed.

\noindent Leaving cultural education aside, researchers debate the effect of racial awareness, if any, on educational outcomes. Some assert that African American adolescents who recognize the systematic race-based inequality in economic and social opportunities see education as offering little help with future life and occupational pursuits. Taylor et al. (1994) support this hypothesis using a sample of 344 African American and White students attending public and Catholic high schools. As African American students become more aware of racial discrimination in American society, their motivation, engagement and effort in school declines. Studies by Hanushek et al. (2009) and Hoxby (2000) discussed above also confirmed this notion that a higher percentage of black schoolmates reduces achievement for blacks.

\noindent Other researchers argue that group affiliation and awareness of past and contemporary social inequities can be motivational. By interviewing 46 low-income African American students who attended two public high schools in Chicago, O’Connor (1999) finds that this awareness creates a strong interest in learning and motivates students to perform better at school. This argument that racial identity positively affects educational outcomes for African American adolescents is supported by Akerlof and Kranton (2002) where they presented that being in the same class with peers from one’s own race has positive effects on black student’s academic performance. By putting forward a conceptual framework of counting racial identity into economic analysis, In their model, Akerlof and Kranton (2002) suggested that schools would make adjustment so that more students would identify with the school. As a result, schools with a majority of black students would establish an identity model that conforms to their black students, and vise versa. Therefore, it’s more likely for black students to “dread” English and math classes in a white-majority school than in a black-majority school, with the opposite for white students. This is consistent with the argument that identifying with one’s ethnic identity can positively affect black students’ educational outcomes.

\noindent Interestingly, what Taylor et al. (1994) also found is that there exists no relationship between African American students’ awareness of discrimination and their self-perception of their ethnic identities, and students’ ethnic identity was actually positively associated with their school achievement and engagement. Therefore, in this sense racial identity theory can explain both sides of the own race peer effects discussion. The positive own race peer effect is based on cultural pedagogy and the education of racial identity, as what Akerlof and Kranton (2002) studied, otherwise it would only be counterproductive, as Hoxby (2000) proposed. 

\noindent This view can be confirmed by Chavous et al. (2003)’s study. By introducing a “Profile Approach” as a guide for the model, Chavous et al. (2003) distinguished factors other than relationships between particular racial identity components and academic outcomes and labeled four clusters. Their chi- square analysis result indicated that compared with \textbf{buffering/ defensive} (individuals who held positive group beliefs and felt that society did not value African Americans) and \textbf{idealized} (people who had strong and positive group affiliation and felt that society valued African Americans) identity groups, \textbf{alienated} (individuals who had low connection to African Americans, felt that African Americans were devalued by society, and therefore felt negatively about them) racial identity group not only indicated less interest in school, but also had a higher percentage of individuals who are chronically absent. When it comes to post-secondary education alienated group again has the lowest percentage of individuals attending these institutions. Therefore, as the African American identity is something that African Americans cannot easily escape in the society where that identity has a major effect on their life chances, they can have different relations with their identity and that would make a difference in their life outcome. As not all black kids can be identified as “idealized”, intervention is necessary to help African American students succeed. 

\noindent\textbf{Cultural Relevant Pedagogy}

\noindent One promising intervention would be Cultural Relevant Pedagogy (CRP). Compared with traditional pedagogy, CRP specifically commits to collective empowerment of underrepresented groups like black students by providing a positive racial identity. Researchers believe that the alienating school and classroom experience for marginalized students can be alleviated by a curriculum that enhanced knowledge of African-American culture and a classroom experience that creates a positive ethnic identity (Banks, 1991; Gay, 1988; Ladson-Billings, 1992; Nasir & Saxe, 2003; Valenzuela, 1999). Therefore, CRP not only focuses on academic success, but also devotes in cultural competences and critical consciousness on cultural norms and values that produce and maintain social inequities (Ladson‐Billings, 1995), and advocates for instructional environments that focuses on validation and affirmation of cultural identities and intellectual capacities of marginalized students (Gay, 2010). 

\noindent One example of classroom practice of cultural competence would be from Gertrude Winston, a white female teacher (Ladson‐Billings, 1995). She involved parents of her students in the classroom as carpenters, professional basketball players, nurses, and church musicians. Through the first-hand experience, she helped set up the role model effect for student and reinforced the idea that their parents were knowledgeable and capable resources. As a result, her students learned that what they had and where they came from was of value. On top of that, students are expected to develop a broader sociopolitical consciousness and be able to critique the cultural norms, social values, and structural inequities, which is the critical consciousness part in CRP advocates for. For example, some teachers encouraged students to critique controversial ideas represented in textbooks and brought in articles and newspapers stating counter-knowledge to help students develop different perspectives on a variety of social and historical phenomena.

\noindent One empirical study supports the ability of CRP to reduce the achievement gap. Dee and Penner (2017) use a quasi-experiment in the San Francisco Unified School District to measure the effect of CRP on attendance and the GPA of students whose 8th grade GPAs are below a certain threshold. Their regression discontinuity design is based on a year-long ninth-grade Ethnic Studies (ES) course implemented by three high schools that only assigned some of their ninth graders to this course. The results suggest that CRP dramatically improves the educational outcomes of at-risk students: assignment to this course increased 9th grade attendance by 21 percentage points, GPA by 1.4 grade points, and course credits earned by 23. Other studies confirm these results. Byrd (2016) surveyed 315 6th through 12th grade students across the United States. The study finds that culturally relevant courses are associated with a statistically significantly increase in academic outcomes and positive ethnic identity.

\noindent For the rest of the paper, I will examine a program that implemented Cultural Relevant Pedagogy both qualitatively and quantitatively to find out more about to what extent does programs implementing CRP would affect student outcome and whether the effect of this kind of pedagogy would have different influence on different targeted age groups.

