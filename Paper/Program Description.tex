The Manhood Development Program (MDP) was launched along with the African American Male Achievement (AAMA) initiative back in 2010. It was first launched in a few high schools in the district and gradually expanded to middle schools and elementary schools. The detailed program implementation year is summarized in the following table. 

\begin{table}[H]
  \begin{center}
    \caption{Manhood Development Program Implementation Year}
    \label{tab:table1}
    \renewcommand{\arraystretch}{1.5}
    \begin{tabular}{l c c} % <-- Changed to S here.
      \hline
      \textbf{School Name} & \textbf{School Level} & \textbf{Implementation School Year}\\
      \hline
      Oakland High & High & SY 2011-12\\
      Oakland Technical High School & High & SY 2011-12\\
      Skyline High School & High & SY 2011-12\\
      McClymonds High School & High & SY 2011-12\\
      Dewey Academy & High & SY 2012-13\\
      Montera Middle School & Middle & SY 2013-14\\
      Piedmont Elementary School & Elementary & SY 2013-14\\
      Claremont Middle School & Middle & SY 2014-15\\
      Redwood Heights Elementary & Elementary & SY 2017-18\\
      Korematsu Elementary & Elementary & SY 2019-20\\
      Sankofa Elementary & Elementary & SY 2019-20\\
      \hline
    \end{tabular}
  \end{center}
\end{table}

MDP was created in response to the fact that past initiatives had little effect in transforming the educational attainment, academic success or school experience of African American male students in the district. Although Black students were offered these “resources”, the systematic racism, low expectations and marginalization in school have blocked them from succeeding. (Watson, 2014)

MDP, aimed as a novel academic mentoring model designed and implemented by African American males for African American males, offers an elective course during the school day taught by African American males for African American males. Black male instructors are chosen based on their understanding of black youth achievement, cultural competency, and their past teaching experience. Teacher races has been a topic that is very controversial: Because of the racial segregation history in the US, mandating segregated teachers is neither advisable or legal. However, truth is instructors in many urban school districts do not match the racial or economic backgrounds of their students and hence would have trouble understanding what students actually need. While racial identity of teachers creates challenges, it also brings opportunities, as “finding staff who can identify with students and provide a model for them can help foster trust and strong relationships.” (Knudson, 2016). The theory of the demonstration effect (role model effect) of black teachers is confirmed by Gershenson et al. (2018) by studying Tennessee's Student Teacher Achievement Ratio (STAR) project. Their results show that black students randomly assigned to a black teacher in grades K-3 are 5 percentage points (7 percent) more likely to graduate from high school and 4 percentage points (13 percent) more likely to enroll in college than their peers in the same school who are not assigned a black teacher. Therefore, having same race teachers for black students would be motivating for black students not only because those teachers can act as role models but also because they might know better what students want. 

Given these evidence, CRP can serve as a way to provide some of the same positive effects as same-race teachers. According to an interview with Brother Abdel-Qawi (Watson, 2014), he mentioned the most important trait he, as Oakland High (one of the first three schools to join the program) Principal, looked for in MDP teachers is patience. This is followed by preference for ones whose college major was African American Studies, who have worked with African American male organizations in the past, who demonstrated their connectedness to the black community through art, faith-based work, etc., and who were connected to African American boys specifically not only on a professional level but also on a personal level. In general, MDP look for black instructors who “not see their position as merely a job, but a calling”. 

Those instructors in MDP would design an elective course offered every Monday through Friday during the school day. A 20-25 students’ cohort is made up heterogeneously with one third of the class who demonstrated academic success, one third being average, and one third who are under-achieving. In the elective course, students are taught culturally relevant contents. For example, a high school freshman course, Mastering Our Identity, covers different topcs like Ma’at, an analysis of ancient African civilizations; Maafa, the African American Holocaust; and Sankofa, the struggle for liberation and dignity. The class engages students by projects where students are given the chance and responsibility to discuss and design interventions to empower African Americans. These classes adopt an anti-oppression pedagogy that emphasizes open-ended questions, group discussions, cultural consciousness building and critical thinking skills. The aim of CRP is for students to have a better understanding of where they stand as African Americans in the society: academically, culturally, emotionally, and socially.

MDP offered the following resources to students in the program: (1) effective African American male instructors; (2) a curriculum that implements Culturally Relevant Pedagogy (CRP); (3) activities to develop leadership and character; (4) College/career guidance and transcript evaluation; (5) parent training and community building; (6) cultural, college and career field trips. The central idea of the program is to engage, encourage, and empower. The essence of MDP is to not only create a safe space for African American boys on campus, provide them with opportunities to experience life outside of school, but also to offer African American male mentors and educators and to utilize a culturally relevant curriculum to learn about their identities and the legacies that they come from.