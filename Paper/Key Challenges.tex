%Finally, highlight the key challenges that you are having to address…This part will be important for discussion during your oral presentation.

How to lessen racial inequality in education has long been a question researchers and policy makers try to figure out. This paper looked back to past initiatives that have been taken to closer the racial achievement gap, and explored the new promising approach: Cultural Relevant Pedagogy (CRP). 

Fixed effects regression show whether or not enrolled in the Manhood Development Program (MDP) has no statistical significant difference on black students' math percentage of proficiency. While the synthetic control approach does show a positive effect of MDP on black students' math percentage of proficiency on the elementary and middle school level, I would be rather careful to draw a causal conclusion because my data doesn't show what happened between 2014 and 2016, so there might be other factors other than the program affecting student performance.

Although school level evaluation can't provide an effective support that the implementation of CRP can narrow the racial achievement gap, according to the interview and survey on parents and students who are involved in the program, it has been a great success on helping black male students rebuild their self-esteem and their values of themselves and the world around them. 

It's debatable whether CRP is an effective approach to help lift up black male students. For one, students' success shouldn't be merely evaluated by their test scores; What's more, CRP may offer students a life-long impact, which is not captured by the measure in this paper. According to one of the MDP instructors brother Jahi (Watson, 2014), “We are not teaching Math or Science or English, those are just the subjects. We are first and foremost teaching human beings.” Success is not necessarily a clear result, but a lifelong process.