There are a few limitations of both the data and the two models I presented in the previous sections.

First of all, data on the dependent variable, math percentage of proficiency, provided by EdFacts from the US Department of Education, is inaccurate to protect students' personal information. Therefore, I can only drop the censored 'PS' cells and take the average of the ranges that are blurred, and this may lead to inaccurate fixed effects regression results and synthetic control model plots. In my future research, I will try to access the restricted student level data and perform a more accurate analysis. Another approach can be the Niche database, of which the data is only accessible to licensed users. However, this database may have good academic performance indicators on the school level and also has more control variables.

Second, in the fixed effects model, the potential Omitted Variable Bias (OVB) problem can be a threat to the internal validity. Although fixed effects regressions control for all unidentifiable variables that vary across schools but not over time, or any variable that varies over time but not across schools (In this case I examined in Stata using the "testparm" command - since the F-stats is > 0.05, I failed to reject the null that the coefficients for all years are jointly qual to zero, therefore no time fixed effects are needed), it can't control for unidentifiable variables that vary both across schools and over time. This is also why I used the synthetic control models after that. However, this leads to the third problem.

Third, a problem with the synth control method is that my data has a gap between years, and too few years in both the pre-treatment period and post-treatment period make the model less convincing in terms of establishing a causal relationship. Instead of having years 2011 to 2017, I only have 2011, 2013, 2015, 2017. This is because the critical control variable \textit{out of school suspension rate} from the CRDC database only provides data from every other years. Nevertheless, similar to the first problem, this problem can be solved if I can get access to the restricted Niche database and replace it with another control variable that is of high collinearity with the \textit{out of school suspension rate} variable.

Regardless of the above limitations, I do see some evidence that suggests the positive effect of CRP on the academic performance of black students. At the very least, even though quantitative evidence can't prove that the program is beneficial in terms of raising black student's test scores, it is not a reason to invalidate the program. I will elaborate this in the discussion section.