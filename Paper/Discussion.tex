My study, unlike other empirical studies that have shown that CRP is a practical approach to closer the achievement gap, doesn't find compelling evidence that suggest the Manhood Development Program (MDP) in Oakland Unified School District (OUSD) have a positive effect on raising the academic performance of poor black students. However, besides the limitations above, there are two reasons that may bias the study downward.

First, every MDP class is made up of a third of under-performing students, a third of average students, and a third of over-performing students. What's more, a report by Watson (2014) clearly states that this program disproportionately helps under-performing students. Even though the academic performance of this group of students may be improved as a result of MDP, their test scores may still haven't reached the threshold of "proficient", which is why we can't see a huge effect of the program when using the math percentage of proficiency as the response variable - some of the improvement may not have been picked up by the dependent variable. Similarly, the improvement of academic performance for high-achieving students might not be picked up by the dependent variable either: their math scores are already above the proficiency threshold before the treatment, and even though their math scores have been improved, the improvement isn't shown on data. Therefore, there are sensitivity issues on both sides and the effect of the program is likely underestimated.

Second, as Dee and Penner (2019) found, MDP reduces the dropout rate for black students in the high school. If this is true, then it can create a potential downward bias for this study. Low-performing students tend to drop out of high school, but with MDP lowering the dropout rate, now those who were to drop out actually stayed in school and although their test scores might have been improved by the program, the overall math performance is lower. Therefore, this is a potential reason why we can't see the program improving students' test scores. Since now the sample selection would be different and the sample size is actually larger, there exists a sample selection bias, causing the effect of the program to be biased downwards.

Nevertheless, even though MDP really does have zero effects on test scores, it wouldn't necessarily be a reason to invalidate the program and the Cultural Relevant Pedagogy since the curriculum emphasizes things other than the tested subjects. Rather, it gets us thinking: what is the real purpose of schooling? Is standardized test scores the only or right way to evaluate the success of a pedagogy and educational effectiveness? According to Popham (1999), "Employing standardized achievement tests to ascertain educational quality is like measuring temperature with a tablespoon." The purpose of MDP and CRP is to empower underrepresented students by developing cultural competence and critical consciousness on cultural norms that produce social inequalities, and this mindset will benefit students more than test scores later in life. Although students in the program come from different academic backgrounds, they form a bond of brotherhood inside MDP classrooms and motivate and learn from each other. All-black schools are not new and they were even thought of as a form of racism and therefore inappropriate in the past. However, creating an identity and gender-specific program within a diverse large urban school district is groundbreaking. Additionally, the program provides students with motivated and devoted black male teachers. By serving as role models, understanding black boys’ struggles, listening to students’ voices, and conveying them to the school and outside world, black teachers in the program provide a safe space for black boys to be themselves and thrive, without worrying about the systematic racism and marginalization from the outside world. More importantly, cultural relevant pedagogy is an effective way for black boys to foster positive self-esteem and critical thinking for school, life, and the whole society. Through these classes, students learn about the invaluable legacy their ancestors left them, and therefore realize the power of personal and collective agency to help them overcome social barriers to succeed. Therefore, it's obviously not comprehensive enough to evaluate this program merely by test scores.

However, programs like this may face challenges as it highly depends on teacher effectiveness. Young (2010) reports the results of an exploratory study of CRP in urban schools using interviews, classroom observations, participant reflections, administrative documents, and online discussions. It found significant variation in what teachers consider as “cultural relevant” teaching, in what they expect students to know, and the topics considered important, and that these differences reflect the teachers’ cultural biases. As an example, Amy, the only non-White teacher in the study, is the only one who is willing to engage her students in discussions on race and racism. Sleeter (2012) argues that most teachers have a faulty and simplistic conception of what CRP is. Other issues affecting the success of CRP include the nature of racism in school settings and the support provided to implement the program.

Furthermore, there are more interesting research questions about CRP that is yet to be answered. Does same race teacher play a role in the classroom? How does the program influence students in different school levels?

I would further investigate the effectiveness of the program on different school levels. According to some researchers,  “The early educational experiences of African American boys are by far the most important in the developmental trajectory of achievement throughout school.”(James Earl Davis, 2008) Although this study can't provide a strong support to the theory, I plan to do another study in the future on the program's impact on different school levels, when more post-intervention years are available. I think this is as important as whether CRP is effective since it can help give policy makers a clearer idea when is the best time should they direct schools to implement CRP, hence bringing out a more efficient outcome.

Another potential research direction would be on long-term effects of these programs. Deming’s (2009) finding about Head Start, a program of the US Department of Health and Human Services that provides comprehensive early childhood education, health, nutrition, and parent involvement services to low-income children and families, indicates that the short-term effects would fade out, especially for black people. When Chetty (2018) was interviewed about education and neighborhood, he mentioned that even we compare white man and black man who come from similar family income backgrounds and live in the same neighborhood, black men have much higher rates of downward mobility than white man. It’s more likely for black people to fall back to poverty and even if they climb up the income ladder in one generation, they tend to fall back down in the next generation. Therefore, I’m curious about whether the nature of the MDP program (that focuses on self-esteem boost and cultural awareness) can bring a change. Although most programs that involved CRP haven't been implemented for long, future researchers should study the long-term effects of MDP and I believe the result will be important for sociologists and economists who are interested education inequality and social segregation. Therefore, future research needs to be done to further explore the CRP approach.