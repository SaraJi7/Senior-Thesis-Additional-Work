Other than the systematic racism deeply rooted in the society, \footnote{Many studies have verified that black students are thought as lesser than white or Asian students in the classroom (McKown and Weinstein, 2008; Howard, 2013).} the image of how black male students perceive themselves plays a role in the racial achievement gap as well. Racial discrimination in education is thus intensified in this vicious cycle. Oakland Unified School District (OUSD) is a representative example of such. 

Before launching the program, black male students in the school district were twice as likely to be chronically absent from elementary, middle, or high school than average. Their standardized test scores are than their white counterparts both in reading and math, with a 39 percent lower score on the English portion of California state exams in school year 2009-10. What’s more, during school year 2010-11, they were five times more likely to be suspended than other students. (Klivans, 2014) Therefore, in 2011, Tony Smith, the superintendent in the school district decided that this is the time to call an end. In his words, “Enough is enough.” 

The Manhood Development Program, unlike any other regular programs that merely provides funding or forcefully push students into another classroom, aims to help black male students not only just survive in a racist world, but also thrive with the tools to transform themselves, their communities, and the society. The program seeks to create a safe space for black male students in school, recruit black male mentors and teachers as role models, offer a cultural relevant curriculum to help students learn who they are and what legacies that they come from, and provide opportunities to experience life outside of school.

Qualitative evidence like interviews and surveys suggest that MDP has been a success: it seems to have achieved the goals it set out for. A thorough report by Watson (2014) included interviews and surveys with program officials, parents, and students shows that the program has received great progress ever since it was launched. In a survey towards MDP students conducted by Watson, 64$\%$ of students report that their MDP peers are like a family to them, 82$\%$ report that MDP activities make them feel proud to be a black male, and 79$\%$ report that MDP makes them want to be successful in school. An MDP student wrote in the survey, “Our Manhood Development class helps us be more respectful to each other and to stay together as a brotherhood. I learned about my heritage and about myself as an African American student.” Knudson (2016) also interviewed one participant, of whom described the effect as “Before high school, I wasn’t comfortable with who I was. I was ashamed of who I was as a black kid … The one thing that still sticks with me is that there’s nothing wrong with who I am. There’s even greatness in my identity, and I’m really grateful for that.” What’s more, Watson’s survey shows that as opposed to the traditional stereotype of lower expectations on black students, students now report that they are being treated with respect, and they are asked to hold each other accountable to a higher standard. During Watson’s interview with students, more people begin to not be afraid of “acting white” and they think that MDP “made it cool to be black and smart”. 

The Program Manager Brother Jahi told Watson a story during their interview about a student who had a break-down in his class and broke the language role due to some PTSD. Brother Jahi moved the boy out of the MDP class because he refused to apologize and said that he “didn’t care”. However, one week later the boy was eager to rejoin the class – he admitted that he “missed the brotherhood and mentorship”. Now, this boy become one of the model students of his class. What’s more, during Watson’s interview with Nicole Wiggins, the parent of a student in the MDP program, she said, “I know my son wanted to do better in school. After signing up for MDP he began to take responsibility for his actions and he changed. Now, he’s making plans for his future.”

Promising results have also been found quantitatively on the program. Dee and Penner (2019) studied the effect of MDP on high school students in the Oakland Unified School District (OUSD) in California. The study compared the effect of assignment with non-assignment to the program. The findings show that MDP significantly reduced the number of black male dropouts, particularly in the 9th grade. When more black students stay in school, I would assume that they are performing better as well. Therefore, in the next section, I want to examine whether this positive qualitative result and the quantitative reduction in dropout rate can be confirmed by student's increased test scores on different school levels.